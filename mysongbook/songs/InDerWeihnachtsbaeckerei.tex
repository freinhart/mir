\songsection{In der Weihnachtsbäckerei \hfill \normalfont Rolf Zuckowski}
\noindent\rule{\columnwidth}{1pt}

\begin{multicols}{2}
\begin{lstsong}
Refrain:
G        (C         G)    Am
   In der Weihnachtsbäckerei
         (F     C)   G
gibt es manche Leckerei.
        (Dm       G)
Zwischen Mehl und Milch
        (Dm      G)
macht so mancher Knilch
    (C     G)   (Dm    G)
eine riesengroße Kleckerei.
      (C         G)   (Am
In der Weihnachtsbäckerei,
F)    (C         G)     C
in der Weihnachtsbäckerei.

Verse 1:
C            C
Wo ist das Rezept geblieben
Dm                 Dm
von den Plätzchen, die wir lieben?
G             G*     G* C*
Wer hat das Rezept - verschleppt?
C               C
Na, dann müssen wir es packen,
Dm                Dm
einfach frei nach Schnauze backen.
G                 G*   G*  C*
Schmeißt den Ofen an - und ran.

Refrain

Verse 2:
C                  C
Brauchen wir nicht Schokolade,
Dm           Dm
Honig, Nüsse und Sukkade
G               G*      G*  C*
und ein bißchen Zimt? - Das stimmt!
C                C
Butter, Mehl und Milch verrühren,
Dm                   Dm
zwischendurch einmal probieren
G                  G*    G* C*
und dann kommt das Ei: - Vorbei!
\end{lstsong}\vfill\columnbreak\begin{lstsong}
Refrain

Verse 3:
C                   C
Bitte mal zur Seite treten,
Dm                Dm
denn wir brauchen Platz zum Kneten.
G               G*      G* C*
Sind die Finger rein? - Du Schwein!
C                   C
Sind die Plätzchen, die wir stechen,
Dm               Dm
erst mal auf den Ofenblechen,
G          G*          G* C*
warten wir gespannt: - Verbrannt!

Refrain:
G        (C         G)    Am
   In der Weihnachtsbäckerei
         (F     C)   G
gibt es manche Leckerei.
        (Dm       G)
Zwischen Mehl und Milch
        (Dm      G)
macht so mancher Knilch
    (C     G)   (Dm    G)
eine riesengroße Kleckerei.
      (C         G)   (Am
In der Weihnachtsbäckerei,
F)    (C         G)    (C* G* C* -)
in der Weihnachtsbäckerei.
\end{lstsong}
\end{multicols}
\newpage

\begin{comment}
       G         D    Em

In der Weihnachtsbäckerei 

        C      G     D

gibt es manche Leckerei.    

         Am       D

Zwischen Mehl und Milch  

         Am       D

macht so mancher Knilch    

     G           D      Em

eine riesengroße Kleckerei. 

       G         D     G

In der Weihnachtsbäckerei  

Em     G         D     G

in der Weihnachtsbäckerei. 

 

G

Wo ist das Rezept geblieben

Am

von den Plätzchen, die wir lieben?                      

D                             G

Wer hat das Rezept ….....verschleppt?

G

Na, dann müssen wir es packen,

Am

einfach frei nach Schnauze backen.                      

D                             G

Schmeißt den Ofen an      und ran.

 

 

G                D    Em

In der Weihnachtsbäckerei 

        C      G     D

gibt es manche Leckerei.    

         Am       D

Zwischen Mehl und Milch  

         Am       D

macht so mancher Knilch    

     G           D        Em

eine riesengroße Kleckerei. 

       G         D     G

In der Weihnachtsbäckerei  

Em     G         D     G

in der Weihnachtsbäckerei. 

 

 

G

Brauchen wir nicht Schokolade,

Am

Honig, Nüsse und Sukkade                                

D

und ein bißchen Zimt?

G

Das stimmt!

G

Butter, Mehl und Milch verrühren,

Am

zwischendurch einmal probieren                  

D                               G

und dann kommt das Ei:       Vorbei!

 

 

G                D    Em

In der Weihnachtsbäckerei 

        C      G     D

gibt es manche Leckerei.    

         Am       D

Zwischen Mehl und Milch  

         Am       D

macht so mancher Knilch    

     G           D        Em

eine riesengroße Kleckerei. 

       G         D     G

In der Weihnachtsbäckerei  

Em     G         D     G

in der Weihnachtsbäckerei. 

 

 

G

Bitte mal zur Seite treten,

Am

denn wir brauchen Platz zum Kneten.                     

D

Sind die Finger rein?

G

Du Schwein!

 

 

G

Sind die Plätzchen, die wir stechen,

Am

erst mal auf den Ofenblechen,                           

D

warten wir gespannt:

G

Verbrannt!

 

 

G                D    Em

In der Weihnachtsbäckerei 

        C      G     D

gibt es manche Leckerei.    

         Am       D

Zwischen Mehl und Milch  

         Am       D

macht so mancher Knilch    

     G           D        Em

eine riesengroße Kleckerei. 

       G         D     G

In der Weihnachtsbäckerei  

Em     G         D     G

in der Weihnachtsbäckerei. 
\end{comment}
